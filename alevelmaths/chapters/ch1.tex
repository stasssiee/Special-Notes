\documentclass[../alevelmaths.tex]{subfiles}
\graphicspath{{\subfix{../figures/}}}
\begin{document}
\chapter{Proof}
\section{Proof}
\subsection*{Introduction to Proof}
In this section we will working with these topics:
\begin{itemize}
    \item Consequence and Equivalence
    \item Proof by Exhaustion
    \item Proof by Deduction
    \item Disproof by Counter-Example
    \item Proof by Contradiction
\end{itemize}

When we look at consequence, we essentially say that "$a$ implies $b$", or:
\[a\rightarrow b\]

If the arrow points the other way, we say that "$b$ implies $a$", or:
\[a\leftarrow b\]

Let's say that statement $a$ states that $p$ is a prime number $>2$.

Let's say that statement $b$ states that $p$ is an odd number.

For these statements, we see that $a$ does imply $b$, so we can write that
\[a\rightarrow b\]
The other way however does not work, since because $p$ is an odd number, it does not imply that $p$ is a prime number.

However, if this was true, we can write that $a$ implies $b$ and $b$ implies $a$, or:
\[a\leftrightarrow b\]
which is sometimes written as "$a$ if and only $b$" or "$a$ iff $b$".

Let's show a logical equivalence. Let $a$ be the statement $n^2$ is odd and $b$ be the statement $n$ is odd.

We know that when $n^2$ is odd, that $n$ is odd when we list out the odd squared numbers. We can see the converse is true as well in this statement since every time a number $n$ is squared, we are given an odd number, therefore:
\[a\leftrightarrow b\]
\subsection*{Proof by Exhaustion}
\subsection*{Proof by Deduction}
\subsection*{Disprove by Counter-Example}
\subsection*{Proof by Contradiction}

\end{document}